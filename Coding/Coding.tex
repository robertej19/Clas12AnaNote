\section{Coding notes}
\href{https://github.com/drewkenjo/dst_monitoring/wiki/Syntactic-sugar}{Syntatic Sugar}


\section{How to Enable Multi-Threading}

    \subsection{1}
        At ifarm, check if you have ~/.groovy  directory
        Otherwise create one.

    \subsection{2}
        Download Andrey’s sugar.
        
        \begin{lstlisting}
            cd ~/.groovy
            wget https://github.com/drewkenjo/dst_monitoring/releases/download/v0/sugar.jar
        \end{lstlisting}


    \subsection{3}
        The standard way to access coatjava at ifarm is adding following commands at \comtilde/.cshrc or \comtilde/.bashrc depending on which shell you use.
        
        \begin{lstlisting}
        source /group/clas12/packages/setup.csh
        module load clas12/pro
        \end{lstlisting}
        
        However, this doesn’t allow you to edit bin files.
        Simply copy run-groovy at any directory with your permission, say \comtilde/.groovy, i.e.
        
        \begin{lstlisting}
        cp $COATJAVA/bin/run-groovy ~/.groovy/
        \end{lstlisting}
        
        edit \comtilde/.groovy/run-groovy ’s line 59 from -Xmx2048m to -Xmx4096m. This will increase memory that can be used in your ifarm.
        
    \subsection{4}
        I guess you have your own standalone analysis script that runs as a main file.
        The idea is to convert this as a class, and call from other main file.
        This is not necessary for parallel computing, but this helps to run multiple analysis at one time and to maintain scripts tidy.
        
        Actually, groovy supports usage of both class and main likewise python’s ‘ if \_\_name\_\_ == “main” '.
        You can put a closure with the same name of the class. See \href{https://stackoverflow.com/questions/52543035/check-if-groovy-script-is-being-executed-directly}{here}
        
    \subsection{5}
        I have standalone analysis script epg.groovy (, which is main itself), \href{https://github.com/Sangbaek/analysis_code/blob/analysis/sangbaek/epg.groovy}{here}
        The code’s structure is like this:\\
        line 1–17: import libraries\\
        line 18–127: defining histograms\\
        line 128–319: main loop to read files (ignore line 131–154 because nowadays Nick is placing inbending and outbending in separate directories)
        line 320–384: saving histograms\\
        
        I converted that to a class file dvcs.groovy, \href{https://github.com/Sangbaek/analysis_code/blob/analysis/sangbaek/dvcs.groovy}{here}
        Here’re how-to’s.\\
        1) Add package name at line 1\\
        2) Add  import java.util.concurrent.ConcurrentHashMap at line 19\\
        3) class name at line 22. class dvcs\{ for this case\\
        4) Change histograms to be used for ConcurrentHashMap. Please compare line 24–60 of dvcs.groovy to 18–127 of epg.groovy.\\
        5) add def processEvent(event)\{ before the loop starts (line 80). The loop contents don’t have to differ from your previous main file.
        6) Remove all histogram saving steps (e.g. line 320-384 of my epg.groovy)\\
        7) Instead, you can set up your histogram's name and directory directly when you fill the histogram,
        e.g.) hists.computeIfAbsent("/epg/elec\_polar\_sec"+ele\_sec, h\_polar\_rate).fill(Math.toDegrees(ele.theta()))
        This will save histograms at /epg/elec\_polar\_sec1–/epg/elec\_polar\_sec6
        Andrey’s sugar will deal with directory structures so that you don’t have to "out.mkdir(dir); out.cd(dir); out.addDataSet(hist)” manually.
        
        By the way, with Andrey’s sugar, you can now add or subtract LorentzVector like def VmissG = beam + target - ele - pro at line 109.\\
        
    \subsection{6}
        Now download \href{https://github.com/Sangbaek/analysis_code/blob/analysis/sangbaek/run.groovy}{this}\\
        Change line 22 of run.groovy to your own class name.\\
        To run this,\\
        
        run-groovy sangbaek/run.groovy `find /cache/clas12/rg-a/production/recon/fall2018/torus-1/pass1/v0/dst/train/skim8/ -name "*.hipo”`
        
    \subsection{7}
        
        
        
        As for 3, I forgot to tell you that line 5–7 needs to be changed as
        
        \begin{lstlisting}
        CLARA_HOME=$COATJAVA/bin/..
        CLAS12DIR=$COATJAVA/bin/..
        CLAS12DIR=$COATJAVA/bin/.. ; export CLAS12DIR
        .
        
        \end{lstlisting}
        
        
        
        To use your edited run-groovy, there are two ways that I know of.\\
        
        First way is to include following line at \comtilde/.cshrc.\\
        alias run-groovy /home/sangbaek/.groovy/run-groovy\\
        
        Second one is to include following shebang line at your groovy scripts like below.\\
   
        \begin{lstlisting}     
        #!/home/sangbaek/.groovy/run-groovy\\
        \end{lstlisting}       
        
        I find the first one easier.
