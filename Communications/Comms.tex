\section{Random Notes}
U channel needs central detector

Hi Colleagues: Bobby, Andrey,  et al.:
Getting back to Valery’s initial comments re. the very interesting graphs which were shown for DVMP (pi0) events. They  showed that a large number of  recoil protons are detected in the central detector - even more than in the forward detector. Of course for the transversely GPD calculations, the t dependence is crucial.  Is it possible to show the same type graphs, but as a function of the t variable. Also, even more interesting would be, a two dimensional graphs  showing the distributions of events vs Q2 and t  for a given W, e.g. 3 GeV, for both the forward and central detectors.
Best regards,
Paul Stoler

Thanks for the table.  It is very useful.  I think that if we have plots for t-distribution for q2 vs. x_b bins for different exclusive channels, which are similar to what Bobby showed during the last meeting, things will be more clear.      I think that  t-distribution for each q2 vs. x_b will clearly shows where we are losing.  These plots are similar to what Paul asked the other day but I think this would be good enough since we are analyzing data in q2, x_b, and t as you pointed out.   Thanks.

best regards, Kyungseon

I agree that we still need to look at the phi coverage to see whether that bin is still analyzable.  However, I think that at this point, plots for t-distributions for a given q2 vs. x_b bin will give us enough idea of our data landscape for different channels.  Thanks.


Dear all,

I highly appreciate if everyone who does exclusive channel analysis, makes plots for t-distribution for q2 vs. x_b bins as Bobby showed today with/without central tracking (or forwarding and central tracking separately depending on your channel) to see the impact of central tracking in terms of kinematic coverage.  It would be very useful to know for the collaboration.   Thanks.

Best regards, Kyungseon


Dear All,

I would like to stress the importance of the central detector in the analysis of the exclusive reactions, DVCS and DVMP. We saw today that the number of the DVMP events are at least twice  more when  proton is detected in the central detector.  But it is not only the statistics that will be double with the inclusion of the central detector. These events have small t, and it is extremely important for the GPD physics. The parameter t/Q2 controls the contribution of  the high twists. So, the central detector will supply the most valuable data set from the physics point of view. When we try to extract the GPD parameters from the experimental data (that we are doing right now) the theorists always request to include data only with t/Q2<1, or even t/Q2<<1. This condition is not always satisfied in the CLAS6 kinematics. That is why we want to have fully operational central detector. I hope that we will make another pass1(or pass2 if you want) this year with improved central tracking that will give us the possibility to explore full CLAS12 kinematics.

